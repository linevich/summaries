\chapter[~]{Знайомство з програмою для розробки креслень sPlan}

\textbf{Мета роботи} --- ознайомитись з призначенням та основними прийомами роботи в програмі для
розробки креслень схем sPlan, навчитись виконувати креслення простих схем.


\section{Короткі теоретичні відомості}

\textbf{Програма sPlan} --- простий і зручний інструмент для креслення електронних і електричних
схем, вона дозволяє легко переносити символи з бібліотеки елементів на схему і прив'язувати їх до
координатної сітки.

Програма підтримує автоматичну нумерацію елементів, контактів та можливість задання номіналів для
елементів. Також є можливість створювати власні шаблони документів, а також компоненти та
бібліотеки, що дає змогу створювати набори стандартних компонентів і поширювати їх між розробниками.

В функціонал також входить робота з текстовими елементами та таблицями. Також можна згенерувати
перелік використаних компонентів.

Є можливість задавати розмірні лінії, що може бути зручним для схематичного зображення об'єктів.

Один документ в sPlan може мати декілька сторінок різних форматів, що полегшує роботу з
багатосторінковими елементами.

Якщо в документі є повторювані значення, наприклад шифр документу то його можна задати у вигляді
користувацької змінної, і вставляти автоматично. Це може бути також використано при створенні
шаблонних документів. Створивши шаблон з вбудованим набором змінних отримаємо можливість
автоматичної вставки значень в документ в налаштуваннях.
