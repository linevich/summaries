\chapter{Методи кількісної оцінки ефективності та якості. Автоматизована кількісна оцінка виробничої
ситуації.}

\section{Короткі теоретичні відомості}

При проектуванні технічних систем, об’єктів виробництва тощо часто виникає проблема вибору
найкращого рішення. Одна і та сама ціль може бути досягнута різними способами (різними методами
передачі інформації у випадку систем зв’язку, різними методами конструкторської реалізації
тощо). При цьому певні способи краще будуть вирішувати одні задачі і гірше інші задачі, інші способи
--- навпаки.

Загалом, якщо проблема вибору може бути вирішена багатьма шляхами, виникає необхідність порівняння
варіантів, які обираються або проектуються, або рішень, що приймаються. Для цього необхідно мати
можливість оцінити, які варіанти кращі, які гірші.

Для порівняння варіантів, які оцінюються, користуються поняттям \textit{ефективності}, що є
узагальнюючою характеристикою об’єктів. Під ефективністю часто розуміють головну, визначальну
характеристику або сполучення найбільш важливих техніко-економічних характеристик.

У кібернетиці \textit{ефективність} якісно повинна виражати пристосованість системи до
функціонування за певним алгоритмом і кількісно повинна визначатися функцією, що виражає
співвідношення між метою, що досягається, та витратами.

Для того, щоб вибір був обґрунтований, необхідно мати математичний (кількісний) критерій оцінки
варіантів. Підхід до побудови критеріїв оцінки систем базується на використанні системного
аналізу. При цьому виходять з наступного. Стосовно спостерігача будь-яка система є об'єктом, весь
комплекс показників якого можна розбити на два класи:

\begin{enumerate}
\item показники, збільшення кількісної міри яких спостерігач сприймає як поліпшення властивостей
системи (об’єкта); ці показники посилюють позитивні якості системи (бажані характеристики);
\item показники, збільшення кількісної міри яких спостерігач сприймає як погіршення властивостей
системи; ці показники посилюють негативні якості системи (небажані, шкідливі характеристики).
\end{enumerate}
   
Показники першого класу назвемо позитивними, другого --- негативними.

Отже, показники, збільшення кількісної величини яких сприймається як покращення властивостей
об’єкта, називають \textit{позитивними}. Показники, збільшення кількісної величини яких,
сприймається як погіршення властивостей об’єкта, називають \textit{негативними}.

До позитивних показників відносяться: у випадку технічних систем зв’язку вірогідність і швидкість
передачі інформації, надійність системи тощо, до негативних --- апаратурна складність системи, її
вартість, займана смуга частот тощо; у випадку промислових підприємств, що виготовляють продукцію
--- до позитивних показників відносяться: товарна продукція, сума прибутку; до негативних ---
кількість працівників, матеріальні витрати, основні виробничі фонди тощо.

Коли показник один, то вибрати просто: найкращій варіант той, для якого показник, значення якого
бажано збільшити (позитивний), має більше числове значення, а показник, значення якого збільшувати
небажано, (негативний) --- має найменше значення.

У загальному випадку позитивні і негативні показники є суперечливими і взаємозалежними: збільшення
значення (кількісної міри) позитивних показників може привести до збільшення значення негативних
показників.  Наприклад, для боротьби з перешкодами в каналі передачі інформації часто застосовується
метод уведення надмірності в передану інформацію. Цей метод є ефективним засобом організації системи
з дуже високою вірогідністю правильної передачі. Однак уведення штучної надмірності в передану
інформацію значно ускладнює систему передачі інформації. У цьому прикладі має місце протиріччя між
вимогами до вірогідності передачі інформації й апаратурної складності системи. Протиріччя можуть
мати місце як між позитивними і негативними показниками, так і між позитивними і позитивними (серед
позитивних показників) чи між негативними і негативними (серед негативних показників). Наприклад, у
вище приведеній системі передачі інформації збільшення значення вірогідності передачі (позитивного
показника) шляхом введення штучної надмірності у виді додаткових кодових символів приводить до
зменшення значення іншого позитивного показника - швидкості передачі інформації.

Отже, проблема вибору ускладнюється, коли показників декілька, причому вони можуть бути як
кількісні, так і якісні --- міра зручності користування, наявність чи відсутність певних функцій,
складність побудови, ступінь якості наявної документації. В такому випадку стоїть задача формування
критерію ефективності, що буде виражати узагальнено, в сукупності, характеристики об’єкту. Причому,
необхідно, щоб він враховував кількісну міру як бажаних характеристик, так і небажаних.

Загальні вимоги до критеріїв оцінки систем можна сформулювати таким чином:

\begin{enumerate}
\item кількісний вираз ефективності системи;
\item врахування усіх показників системи;
\item врахування важливості кожного показника;
\item врахування взаємозв’язку між показниками;
\item індивідуальність шуканої функції аналітичної форми запису критерію щодо порівнюваних
  варіантів;
\item явне вираження функції, що описує форму критерію, і її придатність для порівняння варіантів
  системи.
\end{enumerate}
   
Введення понять позитивних і негативних показників дає можливість виразити весь комплекс показників
системи одним відносно простим математичним співвідношенням з урахуванням коефіцієнтів важливості
кожного показника. Дійсно, при такому визначенні показників можна вважати, що головна задача синтезу
оптимальної системи полягає в досягненні найбільшої суми значень позитивних показників і найменшої
суми значень негативних показників з урахуванням важливості кожного показника. Далі природно
припустити, що з усіх варіантів, кожний з який має як позитивні, так і негативні показники з різними
значеннями, найкращим для спостерігача буде той варіант, що задовольняє заданим вимогам щонайкраще.

Таким чином, критерій має бути таким, щоб при загальному збільшенні показників, що виражають функцію
цілі системи, значення критерію зростало, а при зростанні сукупних витрат --- зменшувалося. При
наведеному вище визначенні двох видів показників системи ціль, що досягається, (якість системи)
характеризує сума значень позитивних показників, а витрати - сума значень негативних показників. Усі
ці показники в усіх без винятку випадках можуть бути визначені кількісно.

Отже, весь комплекс показників системи необхідно задати одним сукупним параметром $F_0$ --- функцією
ефективності, що виражається через значення позитивних $\displaystyle\sum_{i=0}{n} X_i$ значення
негативних $\displaystyle\sum_{j=0}{m} Y_j$ показників з урахуванням важливості кож-ного
показника. При цьому сукупний параметр F0 повинний збільшуватися зі збільшенням суми значень
позитивних показників і зменшуватися зі збіль-шенням суми значень негативних показників , де m і n
--- кількість позитивних і негативних показників системи відповідно $(i=1,2,\ldots,n;
j=1,2,\ldots,m)$.

Оскільки кожен показник $Х_i$ і $Y_j$ має різну важливість в системі заданих вимог (зовнішніх умов),
то показники $Х_i$ і $Y_j$ повинні бути перемножені на певні вагові коефіцієнти важливості $a_i$ і
$b_j$ відповідно. Вагові коефіцієнти $a_i$ і $b_j$ виражають кількісну міру важливості позитивних і
негативних показників відповідно.

Таким чином, сукупний параметр $F_0$ можна представити наступною функцією:
\begin{equation}
  \label{eq:sum_param} F_0 = f(\displaystyle\sum_{i=1}^{n} a_i \cdot X_i, \displaystyle\sum_{j=1}^{m}
b_j \cdot Y_j)
\end{equation}
\begin{itemize}[labelsep=*]
\item [Де:] $\displaystyle\sum_{i=1}^{n} a_i \cdot X_i$ --- кількісна величина якості системи;
\item []$\displaystyle\sum_{j=1}^{m} a_j \cdot Y_j$ --- сукупні витрати.
\end{itemize}

Це ж співвідношення виражає конкретне число й охоплює всі показники системи з урахуванням кількісної
міри значень і важливості показників системи.

В даній лабораторній роботі критерій ефективності розглядається як лінійна комбінація кількісних
показників:
\begin{equation}
  \label{eq:quanitiy_factor}
  F_k = \displaystyle\sum_{i=1} a_i \cdot X_{ik} - \displaystyle\sum_{j=1} b_j \cdot Y_{jk}
\end{equation}
\begin{itemize}[labelsep=*]
\item [Де:] $k$ --- номер варіанту системи;
\item [] $a,k$ --- вагові коефіцієнти показників.
\end{itemize}

В принципі, сформувати функцію критерію ефективності можливо також в інший формі, наприклад як
частка сумарних позитивних та негативних показ- ників, але в такому випадку шкала значень функції
вже не буде лінійною:
\begin{equation}
  \label{eq:quanity_factor_alternative}
  F_k = \frac{\displaystyle\sum_{i=1} a_i \cdot X_{ik}}{\displaystyle\sum_{j=1} b_j \cdot Y_{jk}}  
\end{equation}

Якщо прийняти, що вагові коефіцієнти негативних показників є від’ємними числами, тоді лінійний
критерій можна записати просто як суму добутків значень показників на вагові коефіцієнти:
\begin{equation}
  \label{eq:quanity_factor_linear}
  F_k = \displaystyle\sum_{i=1} a_i \cdot X_{ik} + \displaystyle\sum_{j=1} b_j \cdot Y_{jk}
\end{equation}
або просто, об’єднавши позитивні та негативні показники разом:
\begin{equation}
  \label{eq:quanity_factor_simple}
  F_k = \displaystyle\sum_{i=1} a_i \cdot X_{ik}
\end{equation}

Зведемо значення параметрів декількох варіантів об’єктів до наступного вигляду:

\begin{table}[!ht]
\centering
\caption{Приклад представлення даних в табличному вигляді}
\label{t:table_view_exapmle}
\begin{tabular}{c|c|c|c|c|c|}
\cline{2-6}
\multicolumn{1}{c|}{} & \multicolumn{5}{c|}{Показники} \\ \hline
\multicolumn{1}{|c|}{Варіанти систем (об’єктів)} & 1 & 2 & 3 & 4 & 5 \\ \hline
\multicolumn{1}{|c|}{1} & $X_{11} $ & $X_{21} $ & $X_{31} $ & $X_{41} $ & $X_{51} $  \\ \hline
\multicolumn{1}{|c|}{2} & $X_{12} $ & $X_{22} $ & $X_{32} $ & $X_{42} $ & $X_{52} $  \\ \hline
\multicolumn{1}{|c|}{3} & $X_{13} $ & $X_{23} $ & $X_{33} $ & $X_{43} $ & $X_{53} $  \\ \hline
\end{tabular}
\end{table}

Перед формуванням таблиці вихідних даних значення якісних показників мають бути приведені до певних
кількісних значень.  Крім обраної форми критерію, для розрахунку необхідно дотримати певні вимоги:

\begin{itemize}
\item необхідно, щоб кількісна міра (вибір одиниці виміру – метри, кг, кГц) того чи іншого показника
  не давала перевагу одного показника над іншим. Дійсно, важливим є не чисельне значення
  параметрів, а саме співвідношення значень показника між варіантами. Для цього кількісні
  характеристики X та Y для всіх варіантів нормують, ділячи значення кожного показника $X_{і k}$ на
  максимальне значення в даному стовбці по всім варіантам $max_{(k)}~(X_{ik})$.
\item для обрання найбільш ефективного варіанту за результатами розрахунку важливим є не абсолютне
  значення функції ефективності для кожного окремого варіанту, а співвідношення цих значень. Також
  бажано мати шкалу (обмежений діапазон) можливих значень функції ефективності. Тому виконують
  нормування коефіцієнтів таким чином, щоб значення функції знаходилось в певних межах, напр. від 0
  до 1 (за модулем). для цього вагові коефіцієнти нормують:
  \begin{equation}
    \label{eq:normal_factor}
    \displaystyle\sum_{i}|a_i| + \displaystyle\sum_{j}|b_j| = 1
  \end{equation}
\end{itemize}

Визначений вираз критерію ефективності разом із вимогами щодо параметрів, які в нього входять,
становить модель прийняття рішень, яка досліджується в даній лабораторній роботі.

\newpage
\section{Порядок виконання роботи}

\subsection{Визначення показників оцінюваних об’єктів}

В даній лабораторній роботі необхідно виконати кількісну оцінку трьох можливих варіантів об’єктів
(систем) та визначити оптимальний варіант.

Для оцінювання було обрано носії даних типу SSD.

\textbf{SSD} \textit{(англ. SSD, solid-state drive)} --- комп'ютерний запам'ятовувальний пристрій на
основі мікросхем пам'яті та контролера керування ними, що не містить рухомих механічних частин.
\begin{figure}[!ht]
  \centering
  \includegraphics[width=0.7\linewidth]{images/lab1/ssd.jpg}
  \label{f:ssd} 
  \caption{SSD}
\end{figure}

\subsubsection{Моделі і параметри оцінки}

Моделі:
\begin{itemize}
\item Kingston SSDNow
\item KC400Kingston HyperX Fury
\item Kingston SSDNow V300
\end{itemize}

Параметри, за якими оцінюються варіанти:
\begin{itemize}
\item обсяг
\item швидкість зчитування
\item швидкість запису
\item ціна
\item час наробки на відмову
\end{itemize}

Визначаємо нормовані значення цих показників, та в результаті аналізу системи визначаємо значення
вагових коефіцієнтів(на власний розсуд)При цьому позитивним показникам відповідають додатні
коефіцієнти, а негативним --- від’ємні.

Числові значення коефіцієнтів довільні, але в сумі за модулем повинні складати 1 (згідно вимоги
нормування). Результат формування умови задачі наведено в \ref{t:source_data}.


\begin{table}[!ht]
\centering
\caption{Вихідні дані для оцінки ефективності}
\label{t:source_data}
\begin{tabular}{|C{2.4cm}|C{2.0cm}|C{2.6cm}|C{2.2cm}|C{2.1cm}|C{2.6cm}|}
\hline
Варіанти SSD & Обсяг, GB & Швидкість зчитування, Mb/s & Швидкість запису, Mb/s & Ціна, тис. грн. & Час наробки на відмову, млн годин \\ \hline
Kingston SSDNow KC400 & $128$ & $550$  & $450$ & $0.1$ & $1$ \\ \hline
Kingston HyperX Fury & $120$ & $ 500$ & $500$ & $0.5$ & $1$ \\ \hline
Kingston SSDNow V300 & $120$ & $ 450$ &  $450$& $0.4$ & $1$ \\ \hline
Коефіцієнт важливості $a_i$ ($b_j$) & $0.2$ & $0.3$ & $0.3$ & $-0.15$  & $0.05$ \\ \hline
\end{tabular}
\end{table}

\subsection{Нормування значень показників}

Виконуємо нормування значень показників. Для цього для кожного параметра визначити максимальне
значення по кожному показнику і поділити на нього всі значення у відповідному стовпці. Результат
заносимо в таблицю нормованих значень показників

\subsection{Ручний розрахунок значень функцій ефективності}

\subsection{Перевірка правильності розрахунку на EОМ}