\chapter{Визначення метрологічних характеристик використаних приладів}
\section{Визначення метрологічних характеристик електроного ємнісного гігрометра}

Для визначення метрологічних характеристик ємнісного гігрометра було використано
зразковий(еталонний) гігрометр ``Testo 605 H1'' і мультиметр налаштований на вимрювання ємності який
було під’єднано до досліджувагого гігрометра.

Отримано 11 значень вологості і напруги. Результати обрахунків та вимірювань наведені в таблиці \ref{t:metrological_exp}

\input{./py/print_sensitivity_table.tex}

\subsection{Визначення чутливості кожного з результатів}

Знайдемо чутливість для кожного з результатів:

\begin{equation}
  S = \frac{C}{\varphi}
\end{equation}

\begin{itemize}
\item [Де:] $C$ --- ємність мембрани, ~пкФ;
\item []$\varphi$ ---  значення вологості, ~\%.
\end{itemize}

\input{./py/print_sensitivity.tex}


\subsection{Визначення середньої чутливості}

 Розрахуємо середню чутливість:

 \begin{equation}
   S_{\text{ср}} = \frac{\sum S_n}{n}
 \end{equation}

 \begin{itemize}
 \item [Де:] $S_n$ --- чутливість n-го результату спостереження, пкФ/\%;
 \item []$n$ ---  кількість результатів спостереження.
 \end{itemize}

 \input{./py/print_sensitivity_am.tex}


\subsection{Визначення cумарної середньоквадратичної похибки вимірюваня}

\begin{equation}
  \Delta C_{\sum} = \sqrt{\Delta C_1^2 + \Delta C_2^2}
\end{equation}

\begin{equation}
  \Delta \varphi = \frac{\varphi_{max} \cdot \gamma_{\text{гігр.}} }{100}
\end{equation}
\begin{itemize}
\item [Де:] $\varphi_{max}$ --- максимальне значення вологості, \%;
\item []$\gamma_{\text{гігр.}}$ ---  приведена похибка гігрометра.
\end{itemize}

\begin{gather}
  \varphi_{max} = 95 ~\%; \nonumber\\
  \Delta \gamma_{\text{гігр.}} = 3.158 ~\%; \\
  \Delta \varphi = \frac{95 \cdot 3.158}{100} = 3.0 \%. \nonumber
\end{gather}
\begin{equation}
    \Delta C_1 = \Delta \varphi \cdot S_{\text{cep}} = 3 \cdot 7.72 = 23.16 ~\text{пкФ}
\end{equation}
\begin{gather}
    C_{max} = 2000 ~\text{пкФ}; \nonumber\\
    \Delta \gamma_{\text{мульт.}} = 4 ~\%; \\
    \Delta C_2 = \frac{C_{max} \cdot \gamma_{\text{мульт.}}}{100} = \frac{2000 \cdot 4}{100} = 80 ~\text{пкФ}. \nonumber
\end{gather}

\begin{equation}
  \Delta C_{\sum}  = \sqrt{23.16^2 + 20^2} = 83.285 ~\text{пкФ}
\end{equation}


    
\subsection{Визначення приведеної похибки вимірювання}
Визначимо приведену похибку:

\subsection{Визначимо відносну похибку вимірювання:}
Визначимо відносну похибку вимірювання:

\section*{Висновки до розділу 3}
\addcontentsline{toc}{section}{Висновки до розділу 3}

Визначено метрологоічні характеристики електронного ємнісного гігрометра ``Testo 605 h_1'', проведено
ознайомелення з методикою повірки метрологічних приладів.
