\chapter{Вимірювання вологості у виробничому приміщенні}

\section{Вимірюванння вологості прямим методом}

\subsection{Методика вимірюванння}

Для проведення дослідження було обрано гігрометричний психрометр ``ВИТ-1''.  Прилад було
встановлено на висоті 0.5м, що відповідає вимогам до вимірювання показників мікроклімату для сидячих
робочих місць.

\subsection{Результати спостереження}

\begin{table}[ht]
  \centering
  \caption{Результати спостереження при вимірюванні психрометром}
\label{t:results_direct}
\begin{tabular}{| c | c | c | c | c | c |}
\hline
\multicolumn{1}{|C{3cm}|}{Термометри} &
\multicolumn{1}{C{3.5cm}|}{Вимірювані температури,~°C} &
\multicolumn{1}{C{3.5cm}|}{Поправки до температур за паспортом,~°C} &
\multicolumn{1}{C{5cm}|}{Температура після введення поправок,~°C} \\ \hline 
Сухий  & $22.5$ & $-0.15$ & $22.35$ \\ \hline 
Зволожений & $16.1$ & $+0.20$ & $16.3$ \\ \hline 
\end{tabular}
\end{table}

\subsection{Обробка результатів спостереження}

Округлимо покази сухого термометра до цілих: 
\begin{equation}
  t_c = 22.35 \approx 22.4{^\circ}C.
\end{equation}
Знайдемо різницю температур сухого та зволоженого термометра:
\begin{equation}
\Delta t = t_{c}-t_{\text{зв}} = 6.1{^\circ}.
\end{equation}
\begin{itemize}
\item [Де:] $\Delta t$ --- різниця температур, ${^\circ}C$;
\item []$t_{c}$ ---  температура сухого термометра, ${^\circ}C$;
\item []$t_{\text{зв}}$ --- температура зволоженого термометра, ${^\circ}C$.
\end{itemize}

Визначаємо відносну вологість для $t_c$, для чого інтерполюємо значення відносної вологості
за таблицею для $t_c$ від 22 до 23°С --- отримуємо вологість 48\%. При збільшенні температури на 1°С
вологість збільшується на 2\%, а при збільшенні температури на 0.4°С вологість збільшується
відповідно:
\begin{equation}
  t_{\text{c}} + \Delta t  =  48 +  0.4 \cdot 2 = 48.8\%
\end{equation}

Визначаємо відносну вологість для $t_c$ і $\Delta t$, для чого інтерполюємо значення відносної
вологості при різниці показів від 6.0 до 6.5°С. При збільшенні $\Delta t$ на 0.5°С відносна
вологість зменшується на 4\%, а при збільшенні різниці температур на 0.1°С відносна вологість
зменшується на:
\begin{equation}
  \frac{0.1х4}{0.5}=0.8\%
\end{equation}

Отже, вологість $\varphi$ при температурі $t_c$ і різниці температур $\Delta t$, враховуючи
абсолютну похибку психрометра, що складає $\pm 0.2\%$. буде дорівнювати:
\begin{equation}
  48.8 - 0.8 = 48 \pm 0.2\%.
\end{equation}

\section{Вимірюванння вологості непрямим методом}

\subsection{Методика вимірюванння}

Для вимірювання вологості непрямим методом було обрано електронний ємнісний гірометр ``Testo 605
H1''. Прилад було встановлено на висоті 0.5м, що відповідає вимогам до вимірювання показників
мікроклімату для сидячих робочих місць.

\subsection{Результати спостереження}

Результати отримані під час виконання замірів наведено в \ref{t:results_indirect}.
\pagebreak
\setlength{\mathindent}{0pt}
\begin{table}
\begin{tabular}{|c|c|c|c|c|c|c|c|c|c|c|c|c|c|c|c|c|c|c|c|c|c|c|c|c|c|c|c|c|c|}\end{tabular}
\end{table}

\pagebreak

\subsection{Обробка результатів спостереження}

\begin{enumerate}[leftmargin=*]
\item Обрахуємо середнє квадратичне відхилення

  \begin{equation}
    \delta = \sqrt{\frac{1}{N-1} \cdot \displaystyle\sum_{i=1}^{N} ({\varphi_i} -\bar{\varphi})^2}
  \end{equation}
  
  \begin{itemize}
  \item [Де:] $N$ --- кількість проведених дослідів;
  \item []$\varphi_i$ ---  i-тий елемент результатів спостереження;
  \item []$\bar{\varphi}$ ---  середнє арифметичне результатів спостереження;
  \end{itemize}

  \input{./py/print_delta.tex}
  
\item Обрахуємо середнє квадратичне відхилення середнього арифметичного:
  \input{./py/print_delta_ca.tex}

\item Обрахуємо довірчий інтервал результату спостереження:
  \begin{equation}
    \Delta_i = \bar{\varphi} \pm K \cdot \delta
  \end{equation}

  \begin{itemize}
  \item [Де:] $K$ --- коефіцієнт Стьюдента;
  \item []$\delta$ ---  середнє квадратичне відхилення, \%;
  \item []$\bar{\varphi}$ ---  середнє арифметичне результатів спостереження, \%;
  \end{itemize}
  \input{./py/print_observations_confident_interval.tex}

\item Обрахуємо довірчий інтервал результату вимірювання
  \begin{equation}
    \Delta_i = \bar{\varphi} \pm K_{\text{са}} \cdot \delta_{\text{са}}
  \end{equation}

  \begin{itemize}
  \item [Де:] $K$ --- коефіцієнт Стьюдента;
  \item []$\delta_{\text{са}}$ ---  середнє арифметичне середнього квадратичного відхилення, \%;
  \item []$\bar{\varphi}_{\text{са}}$ ---  середнє арифметичне результатів спостереження, \%;
  \end{itemize}
\end{enumerate}
  
\section*{Висновки до розділу 2}
\addcontentsline{toc}{section}{Висновки до розділу 2}

За допомогою електронного гігрометру ``ВИТ-1'' було проведено вимірювання вологості у виробничому
приміщенні.  Отриманні результати свідчать про те, що значення вологості лежить в діапазоні ДІАПАЗОН
--- ДІАПАЗОН.  Таким чином, згідно стандарту СТАНДАРТ, значення вологості для даного виробничоно
приміщення є оптимальними та відповідають нормі.  Вплив мікроклімату на фізичний стан персоналу
мінімальний.
