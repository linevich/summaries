\chapter{Вимірювання вологості у виробничому приміщенні}

\section{Вимірюванння вологості прямим методом}

\subsection{Методика вимірюванння}

Для проведення дослідження було обрано гігрометричний психрометр ВИТ-1.  Прилад було
встановлено на вистоі 0.5м, що відповідає вимогам до вимірювання показників мікроклімату для сидячих
робочих місць.

\subsection{Результати спостереження}

\begin{table}[ht]
  \centering
  \caption{Результати спостереження при вимірюванні психрометром}
\label{t:results_direct}
\begin{tabular}{| c | c | c | c | c | c |}
\hline
\multicolumn{1}{|C{3cm}|}{Термометри} &
\multicolumn{1}{C{3.5cm}|}{Вимірювані температури,~°C} &
\multicolumn{1}{C{3.5cm}|}{Поправки до температур за паспортом,~°C} &
\multicolumn{1}{C{5cm}|}{Температура після введення поправок,~°C} \\ \hline 
Сухий  & $22,5$ & $-0,15$ & $22,35$ \\ \hline 
Зволожений & $16,1$ & $+0,20$ & $16,3$ \\ \hline 
\end{tabular}
\end{table}

\subsection{Обробка результатів спостереження}

Округлимо покази сухого термометра до цілих: 
\begin{equation}
  t_c = 22,35 \approx 22,4{^\circ}C.
\end{equation}
Знайдемо різницю температур сухого та зволоженого термометра:
\begin{equation}
\Delta t = t_{c}-t_{\text{зв}} = 6,1{^\circ}.
\end{equation}
\begin{itemize}
\item [Де:] $\Delta t$ --- різниця температур, ${^\circ}C$;
\item []$t_{c}$ ---  температура сухого термометра, ${^\circ}C$;
\item []$t_{\text{зв}}$ --- температура зволоженого термометра, ${^\circ}C$.
\end{itemize}

Визначаємо відносну вологість для $t_c$, для чого інтерполюємо значення відносної вологості
за таблицею для $t_c$ від 22 до 23°С --- отримуємо вологість 48\%. При збільшенні температури на 1°С
вологість збільшується на 2\%, а при збільшенні температури на 0,4°С вологість збільшується
відповідно:
\begin{equation}
  t_{\text{c}} + \Delta t  =  48 +  0,4 \cdot 2 = 48,8\%
\end{equation}

Визначаємо відносну вологість для $t_c$ і $\Delta t$, для чого інтерполюємо значення відносної
вологості при різниці показів від 6,0 до 6,5°С. При збільшенні $\Delta t$ на 0,5°С відносна
вологість зменшується на 4\%, а при збільшенні різниці температур на 0,1°С відносна вологість
зменшується на:
\begin{equation}
  \frac{0,1х4}{0,5}=0,8\%
\end{equation}

Отже, вологість $\varphi$ при температурі $t_c$ і різниці температур $\Delta t$, враховуючи
абсолютну похибку психрометра, що складає $\pm 0.2\%$. буде дорівнювати:
\begin{equation}
  48,8 - 0,8 = 48 \pm 0.2\%.
\end{equation}

\section{Вимірюванння вологості непрямим методом}

\subsection{Результати спостереження}

Під час проведдення багатократних замірів вологості були отримані результати наедені в
\ref{t:results_indirect}

\begin{table}[ht!]
  \caption{Результати спостереження}
\label{t:results_indirect}
\begin{tabular}{| c | c | c | c | c | c |}
  \hline
  \multicolumn{1}{|C{1cm}|}{№} &
  \multicolumn{1}{c|}{$\varphi_i$,~\%} &
  \multicolumn{1}{C{3.0cm}|}{{$\Delta\varphi_i = \varphi_i - \bar{\varphi}$}} &
  \multicolumn{1}{C{2.5cm}|}{$\Delta{\varphi_i}^2$} &
  \multicolumn{1}{C{2.5cm}|}{Примітка} \\ \hline
  \rownumber & 47.95 & 1,44 & 1,85 & \\ \hline
  \rownumber & 50 & 1,44 & 1,85 & \\ \hline
  \rownumber & 51 & 1,44 & 1,85 & \\ \hline
  \rownumber & 45 & 1,44 & 1,85 & \\ \hline
  \rownumber & 53 & 1,44 & 1,85 & \\ \hline
  \rownumber & 50 & 1,44 & 1,85 & \\ \hline
  \rownumber & 50 & 1,44 & 1,85 & \\ \hline
  \rownumber & 51 & 1,44 & 1,85 & \\ \hline
  \rownumber & 45 & 1,44 & 1,85 & \\ \hline
  \rownumber & 53 & 1,44 & 1,85 & \\ \hline
  \rownumber & 50 & 1,44 & 1,85 & \\ \hline
  \rownumber & 50 & 1,44 & 1,85 & \\ \hline
  \rownumber & 51 & 1,44 & 1,85 & \\ \hline
  \rownumber & 45 & 1,44 & 1,85 & \\ \hline
  \rownumber & 53 & 1,44 & 1,85 & \\ \hline
  \rownumber & 50 & 1,44 & 1,85 & \\ \hline
  \rownumber & 50 & 1,44 & 1,85 & \\ \hline
  \rownumber & 51 & 1,44 & 1,85 & \\ \hline
  \rownumber & 45 & 1,44 & 1,85 & \\ \hline
  \rownumber & 53 & 1,44 & 1,85 & \\ \hline
  \rownumber & 50 & 1,44 & 1,85 & \\ \hline
  \rownumber & 50 & 1,44 & 1,85 & \\ \hline
  \rownumber & 51 & 1,44 & 1,85 & \\ \hline
  \rownumber & 45 & 1,44 & 1,85 & \\ \hline
  \rownumber & 53 & 1,44 & 1,85 & \\ \hline
  \rownumber & 50 & 1,44 & 1,85 & \\ \hline
  \rownumber & 50 & 1,44 & 1,85 & \\ \hline
  \rownumber & 51 & 1,44 & 1,85 & \\ \hline
  \rownumber & 45 & 1,44 & 1,85 & \\ \hline
  \rownumber & 53 & 1,44 & 1,85 & \\ \hline
  & $\bar{\varphi} = \frac{1}{30} \cdot \displaystyle\sum_{i=1}^{30} \Delta{\varphi_i}^2=50$
  & 0 & $\delta = 50$ & \\ \hline

\end{tabular}
\end{table}

\subsection{Обробка результатів спостереження}

Відносна вологість приміщення на початку зміни:

\begin{equation}
  \varphi = \frac{50 + 50 + 51 + 45 + 53}{5} = 49
\end{equation}

де $\varphi$ --- відносна вологість, \%.

Відносна вологість приміщення в середині зміни:

\begin{equation}
  \varphi = \frac{50 + 50 + 51 + 45 + 53}{5} = 49
\end{equation}

Відносна вологість приміщення в кінці зміни:

\begin{equation}
  \varphi = \frac{50 + 50 + 51 + 45 + 53}{5} = 49
\end{equation}

\subsection{Оцінка похибок}