\chapter*{Висновки}
\addcontentsline{toc}{chapter}{Висновки}

В даному курсовому проекті була розкрита тема вимірювання відносної вологості повітрія при оцінці
мікроклімату виробничих приміщень. Були описані прямі методи вимірювання відносної вологості
повітрія, визначені величини мікроклімату виробничих приміщень, що характеризується сукупністю
параметрів повітря у виробничому приміщенні, які діють на людину у процесі праці, на його робочому
місці та у робочій зоні.

Описали його вплив на організм людини та заходи щодо зниження його негативного впливу. Також було
описано будову та принцип дії основних приладів для вимірювання відносної вологості повітрія.  У
відповідності до завдання були проведенні розрахунки похибок відносної вологості повітрія в
виробничому приміщенні прямим методом.

Вимірювання здійснювалися ємнісним гігрометром, та ``традиційним'' психрометром з двома
термометрами. Розраховано похибки вимірювання відносної вологості повітрія. Результати дослідження
показали, що відносна вологість в даному приміщенні лежить в допустимих межах.

В такому випадку можна вважати, що мікроклімат цього приміщення має незначний вплив на здоров’я
людини.  Оптимальні параметри мікроклімату забезпечують комфортні умови у виробничому
приміщенні. Допустимі параметри можуть викликати у людини дискомфортні відчуття. Але змін у стані
здоров'я не буде.
