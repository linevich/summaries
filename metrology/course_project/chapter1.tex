\chapter{Методи вимірювання вологості повітря}

\section{Прямі методи вимірювання}

\subsection{Вимірювання швидкості випаровування при вимірюванні вологості повітря}

Швидкість випаровування вологи збільшується в міру зменшення відносної вологості повітря.
Випаровування вологи, в свою чергу, викликає охолодження конденсованої. Таким чином, температура
вологого об'єкта зменшується.

За різницею температур повітря і вологого об'єкта можна визначити
швидкість випаровування, а значить, і вологість повітря. При цьому треба враховувати той факт,
волога, яка випаровується залишається навколо вологого предмета, і, таким чином, локально
збільшується вологість повітря. Для усунення цього ефекту при вимірюванні вологості застосовують
аспірацію (створюється потік повітря над вологим об'єктом).

На цьому принципі ґрунтуються \textit{психрометри}.

\section{Непрямі методи вимірювання}

\subsection{Вимірюванння ємності  при вимірюванні вологості повітря}

На дві пластини подається змінна напруга, в залежності від кількості водяної пари між пластинами,
змінюється діелектрична проникність та ємність, яка впливає на реактивний опір конденсатора.

На цьому принципі побудовані \textit{ємнісні електронні гігрометри}.

\subsection{Вимірюванння опору при вимірюванні вологості повітря}

В датчику встановлюється полімерна мембрана яка змінює свій опір в залежності від кількіості
поглинутої вологи.

На цьому принципі побудовані \textit{оіпрні електронні гігрометри}. Варто зауважити, що при
вимірюванні вологості електронними давачами потрібно враховувати температуру, оскільки вона впливає
на калібрування приладів.

\subsection{Вимірювання сили натягу при вимірюванні вологості повітря}

Принцип ґрунтується на здатності знежиреної люської волосини змінювати свою довжину при зміні
вологості, цей принцип взятий за основу у класичних \textit{гігрометрах}.
