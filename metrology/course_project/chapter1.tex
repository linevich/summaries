\chapter{Методи вимірювання вологості повітря}

Для вимірюванння відносоної вологості повітря застосовуються виключно непрямі методи оскільки вони є
значно точнішими і зручнішими у викорисатнні.

\section{Непрямі методи вимірювання}

\subsection{Вимірювання швидкості випаровування}

Швидкість випаровування вологи збільшується в міру зменшення відносної вологості повітря.
Випаровування вологи, в свою чергу, викликає охолодження конденсованої. Таким чином, температура
вологого об'єкта зменшується. За різницею температур повітря і вологого об'єкта можна визначити
швидкість випаровування, а значить, і вологість повітря. При цьому треба враховувати той факт,
волога, яка випаровується залишається навколо вологого предмета, і, таким чином, локально
збільшується вологість повітря. Для усунення цього ефекту при вимірюванні вологості застосовують
аспірацію (створюється потік повітря над вологим об'єктом). Вимірюванння непрямими методами.
