\chapter{Прямі та непрямі методи вимірювання фізичної величини}

Для вимірювання вологості використовуються наступні прилади:

\begin{itemize} \tightlist
\item вологоміри;
\item гігрометри;
\item психрометри.
\end{itemize}

\section{Психрометр}

\subsection{Будова}

Психрометр складається з двох термометрів: сухого та обгорнутого змоченим батистом, а також
\emph{психрометричних таблиць} або \emph{психрометричних графіків} (номограм), за допомогою яких
можна визначити абсолютну та відносну вологість повітря. Також під час вимірювання враховується
атмосферний тиск (у випадку якщо відрізняється від нормального), і вносяться поправки які
зазначаються в психрометричній таблиці. Додатково психрометр може бути обладнаний вентилятором для
обдування повітрям простору навколо вологого термометра.

\subsection{Принцип дії}

Швидкість випаровування вологи збільшується в міру зменшення відносної вологості
повітря. Випаровування вологи, в свою чергу, викликає охолодження конденсованої. Таким чином,
температура вологого об'єкта зменшується. За різницею температур повітря і вологого об'єкта можна
визначити швидкість випаровування, а значить, і вологість повітря. При цьому треба враховувати той
факт, волога, яка випаровується залишається в околицях вологого предмета, і, таким чином, локально
збільшується вологість повітря. Для усунення цього ефекту при вимірюванні вологості застосовують
аспірацію (створюється потік повітря над вологим об'єктом)

\subsection{Види}

Розрізняють стаціонарні (станційні), аспіраційні та дистанційні психрометри.

В стаціонарних психрометрах термометри закріплені на спеціальному штативі в метеорологічній
будці. Основний недолік станційних психрометрів --- залежність показів зволоженого термометра від
швидкості повітряного потоку в будці. Основний станційний психрометр --- психрометр Августа.

В аспіраційному психрометрі (наприклад, психрометр Ассмана) термометри розташовані в спеціальній
оправі, яка захищає їх від пошкоджень й теплового випромінювання навколишніх предметів, де вони
обдуваються за допомогою аспіратора (вентилятора) потоком досліджуваного повітря з постійною
швидкістю. При додатній температурі повітря аспіраційний психрометр --- є найнадійнішим приладом для
вимірювання температури і вологості повітря.

В дистанційних психрометрах як чутливі елементи використовуються термометри опору, терморезистори.

При температурах нижчих від $-5^{\circ}C$ для визначення вологості повітря користуються гігрометром.

\section{Вологомір}

\section{Гігрометр}

\subsection{Будова}

Для вимірювання вологості повітря використовуються гігрометри