\chapter*{Вступ}
\addcontentsline{toc}{chapter}{Вступ}

Контроль показників мікроклімату у виробничих приміщеннях, а зокрема відносної вологості,
здійснюється з метою:

\begin{enumerate}
\item Дотримання умов технологічного процесу --- надто низька або надто висока вологість може
  негативно впливати навіть на волого-нечутливі процеси, та псувати обладнанняю.
\item Забезпечення безпечного робочого середовища для персоналу --- не нормовані значення вологості
  та інших показників мікроклімату згубно впливають на здоров'я персоналу, особливо під час
  тривалого випливу протягом робочої зміни.
\end{enumerate}

Для вимірювання відносної вологості застосовується широкий спектр електронних та аналогових
пристроїв від традиційних писхрометрів з двома термометрами, які застосовуються в основному для
контролю показників безпосередньо на робочому місці працівника до складних точних електронних
пристроїв для контролю умов самого виробничного процесу, який може бути частково чи повністю
ізольований від персоналу.

Контроль параметрів мікроклімату, в тому числі і відносної вологості є невід'ємною і життєво важливою
частиною будь-якого виробництва.
