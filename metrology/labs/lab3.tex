\newpage\BorderFirstPage
\chapter[~]{Дослідження характеристик та параметрів цифрового тахометра}

\textbf{Мета роботи} ---  з’ясувати принцип роботи цифрового тахометра та провести експериментальне
дослідження номінальних статичних характеристик та параметрів фотоелектричного датчика.


\textit{Цифровий тахометр ОЦ-000/30} --- це ЗВТ призначений для вимірювання швидкості обертання в обертах за
хвилину.  Відображення результатів вимірювання виконується на світловому табло.  Через програмний
час для індикації вимірювання і цифрової індикації діапазон вимірюваних швидкостей обмежується лише
розрядністю індикації та потрібною точністю, тому що комплектованим датчиком можна вимірювати
швидкість обертання до 1000 об/хв з приведеною похибкою $\pm1,5\%$ . Для розширення діапазону
вимірювань може бути використаний редуктор. Тому коефіцієнт передачі коліс потрібно врахувати в
дослідах.

