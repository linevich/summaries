\chapter{Визначення метрологічних характеристик використаних приладів}
\section{Визначення метрологічних характеристик електроного ємнісного гігрометра}

Для визначення метрологічних характеристик ємнісного гігрометра було використано
зразковий(еталонний) гігрометр ``ВИТ-1'' і мультиметр налаштований на вимрювання ємності який було
під’єднано до досліджувагого гігрометра.

Отримано 11 значень вологості і напруги. Результати обрахунків та вимірювань наведені в таблиці \ref{t:metrological_exp}

\input{./py/print_sensitivity_table.tex}

\begin{enumerate}[leftmargin=*]
    \item Знайдемо чутливість для кожного з результатів:

    \begin{equation}
        S = \frac{C}{\varphi}
    \end{equation}

    \begin{itemize}
        \item [Де:] $C$ --- ємність мембрани, ~пкФ;
        \item []$\varphi$ ---  значення вологості, ~\%.
    \end{itemize}

    \input{./py/print_sensitivity.tex}

    \item Розрахуємо середню чутливість:

    \begin{equation}
    S_{\text{ср}} = \frac{\sum S_n}{n}
    \end{equation}

    \begin{itemize}
        \item [Де:] $S_n$ --- чутливість n-го результату спостереження, пкФ/\%;
        \item []$n$ ---  кількість результатів спостереження.
    \end{itemize}

    \input{./py/print_sensitivity_am.tex}

    \item Розрахуємо похибку ємності:

    \item Визначимо приведену похибку:

    \item Визначимо відносну похибку вимірювання:
\end{enumerate}

\section*{Висновки до розділу 3}
\addcontentsline{toc}{section}{Висновки до розділу 3}

Визначено метрологоічні характеристики електронного ємнісного гігрометра ``ВИТ-1'', проведено
ознайомелення з методикою повірки метрологічних приладів.
